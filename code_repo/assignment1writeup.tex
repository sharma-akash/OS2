\documentclass[onecolumn, draftclsnofoot,10pt, compsoc]{IEEEtran}
\usepackage{graphicx}
\usepackage{url}
\usepackage{setspace}

\usepackage{geometry}
\geometry{textheight=9.5in, textwidth=7in}

% 1. Fill in these details
\def \CapstoneTeamName{			Dream Team}
\def \CapstoneTeamNumber{		57}
\def \GroupMemberOne{			Parker Bruni}
\def \GroupMemberTwo{			Daniel Schroeder}
\def \GroupMemberThree{			Aubrey Thennell}
\def \CapstoneProjectName{		A Scalable Web Application Framework for Monitoring Energy Usage on Campus}
\def \CapstoneSponsorCompany{	Oregon State University Sustainability Office}
\def \CapstoneSponsorPerson{	Jack Woods}

% 2. Uncomment the appropriate line below so that the document type works
\def \DocType{		Problem Statement
				%Requirements Document
				%Technology Review
				%Design Document
				%Progress Report
				}

\newcommand{\NameSigPair}[1]{\par
\makebox[2.75in][r]{#1} \hfil 	\makebox[3.25in]{\makebox[2.25in]{\hrulefill} \hfill		\makebox[.75in]{\hrulefill}}
\par\vspace{-12pt} \textit{\tiny\noindent
\makebox[2.75in]{} \hfil		\makebox[3.25in]{\makebox[2.25in][r]{Signature} \hfill	\makebox[.75in][r]{Date}}}}
% 3. If the document is not to be signed, uncomment the RENEWcommand below
\renewcommand{\NameSigPair}[1]{#1}

%%%%%%%%%%%%%%%%%%%%%%%%%%%%%%%%%%%%%%%
\begin{document}
\begin{titlepage}
    \pagenumbering{gobble}
    \begin{singlespace}
    	%\includegraphics[height=4cm]{coe_v_spot1}
        \hfill
        \par\vspace{.2in}
        \centering
        \scshape{
            \huge Assignment 1 Write Up \par
            {\large\today}\par
			{\large CS444 Fall 2017}\par
            \vspace{.6in}
            {\Large
                \NameSigPair{Rohit Chaudhary}\par
                \NameSigPair{Akash Sharma}\par
            }
            \vfill
        }
        \begin{abstract}
        % 6. Fill in your abstract
        	In this document, we go over the different parts of assignment 1.
					We will go over the steps needed to perform the request actions as
					required by the assignment. Then, we explain the different flags part
					of the qemu command-line statement. Then, we will discuss our
					solution to the concurrency problem answering all the required
					questions. Finally, we include our version control and work logs.
        \end{abstract}
    \end{singlespace}
\end{titlepage}
\newpage
\pagenumbering{arabic}
\tableofcontents
% 7. uncomment this (if applicable). Consider adding a page break.
%\listoffigures
%\listoftables
\clearpage

% 8. now you write!
\section{Log of Commands}

	List of commands: \\
	cd ../../../../scratch/fall2017 \\
	mkdir 27 \\
	cd 27 \\
	git init \\
	git clone git://git.yoctoproject.org/linux-yocto-3.19 \\
	cd linux-yocto-3.19 \\
	git checkout tags/’v3.19.2’ \\
	git remote set-url origin https://github.com/sharma-akash/OS2 \\
	cd ../../../files \\
	source environment-setup-i586-poky-linux.csh \\
	cp bzImage-qemux86.bin ../fall2017/27 \\
	cp core-image-lsb-sdk-qemux86.ext4 ../fall2017/27 \\
	cp config-3.19.2-yocto-standard../fall2017/27/linux-yocto-3.19 \\
	cd ../fall2017/27/linux-yocto-3.19 \\
	mv config-3.19.2-yocto-standard .config \\
	make -j4 all \\
	cd .. \\
	qemu-system-i386 -gdb tcp::5527 -S -nographic -kernel bzImage-qemux86.bin
	-drive file=core-image-lsb-sdk-qemux86.ext4,if=virtio -enable-kvm -net none
	-usb -localtime --no-reboot --append "root=/dev/vda rw console=ttyS0 debug” \\
	Open new window and connect to the server \\
	Gdb \\
	Target remote :5527 \\
	Continue \\

\section{Qemu Command-line Flags}

	\textbf{-gdb tcp::???} Initialize a GDB stub at port ??? \\
	\textbf{-S} Manually start the CPU \\
	\textbf{-nographic} No graphics for Qemu, only command line \\
	\textbf{-kernel} Set the kernel file image \\
	\textbf{-drive} Add a drive, we add from a file \\
	\textbf{-enable-kvm} Enable KVM virtualization \\
	\textbf{-net} Configure network devices, we set none \\
	\textbf{-usb} Enable usb to add usb devices \\
	\textbf{-localtime} Set what the current time is for the kernel \\
	\textbf{--no-reboot} Don't reboot, just exit \\
	\textbf{-append} defines the command line for the kernel \\

\section{Concurrency Write Up}

	lmao

\section{Version Control Log}

\begin{tabular}{ c | c }
	\hline
	e9411b8 &	 Added Makefile, mersenne, some small changes \\
  \hline
  ce34cbb &  removing temp file \\
	\hline
  f12ca65 & initial commit for concurrency code \\
	\hline
  b108951 &  Adding code\_repo \\
  \hline
\end{tabular}

\section{Work Log}

	10/5 - Created the folder for the group, pulled the code and created a repo \\
	10/6 - Built the kernel, ran the VM \\
	10/7 - Started working on concurrency code \\
	10/8 - Implemented mersenne, continued working on other parts of concurrency\\

\end{document}
