\documentclass[onecolumn, draftclsnofoot,10pt, compsoc]{IEEEtran}
\usepackage{graphicx}
\usepackage{url}
\usepackage{setspace}

\usepackage{geometry}
\geometry{textheight=9.5in, textwidth=7in}

% 1. Fill in these details
\def \CapstoneTeamName{			Dream Team}
\def \CapstoneTeamNumber{		57}
\def \GroupMemberOne{			Parker Bruni}
\def \GroupMemberTwo{			Daniel Schroeder}
\def \GroupMemberThree{			Aubrey Thennell}
\def \CapstoneProjectName{		A Scalable Web Application Framework for Monitoring Energy Usage on Campus}
\def \CapstoneSponsorCompany{	Oregon State University Sustainability Office}
\def \CapstoneSponsorPerson{	Jack Woods}

% 2. Uncomment the appropriate line below so that the document type works
\def \DocType{		Problem Statement
				%Requirements Document
				%Technology Review
				%Design Document
				%Progress Report
				}

\newcommand{\NameSigPair}[1]{\par
\makebox[2.75in][r]{#1} \hfil 	\makebox[3.25in]{\makebox[2.25in]{\hrulefill} \hfill		\makebox[.75in]{\hrulefill}}
\par\vspace{-12pt} \textit{\tiny\noindent
\makebox[2.75in]{} \hfil		\makebox[3.25in]{\makebox[2.25in][r]{Signature} \hfill	\makebox[.75in][r]{Date}}}}
% 3. If the document is not to be signed, uncomment the RENEWcommand below
\renewcommand{\NameSigPair}[1]{#1}

%%%%%%%%%%%%%%%%%%%%%%%%%%%%%%%%%%%%%%%
\begin{document}
\begin{titlepage}
    \pagenumbering{gobble}
    \begin{singlespace}
    	%\includegraphics[height=4cm]{coe_v_spot1}
        \hfill
        \par\vspace{.2in}
        \centering
        \scshape{
            \huge Assignment 1 Write Up \par
            {\large\today}\par
			{\large CS444 Fall 2017}\par
            \vspace{.6in}
            {\Large
                \NameSigPair{Rohit Chaudhary}\par
                \NameSigPair{Akash Sharma}\par
            }
            \vfill
        }
        \begin{abstract}
        % 6. Fill in your abstract
        	In this document, we go over the different parts of assignment 1.
					We will go over the steps needed to perform the request actions as
					required by the assignment. Then, we explain the different flags part
					of the qemu command-line statement. Then, we will discuss our
					solution to the concurrency problem answering all the required
					questions. Finally, we include our version control and work logs.
        \end{abstract}
    \end{singlespace}
\end{titlepage}
\newpage
\pagenumbering{arabic}
\tableofcontents
% 7. uncomment this (if applicable). Consider adding a page break.
%\listoffigures
%\listoftables
\clearpage

% 8. now you write!
\section{Log of Commands}

	List of commands: \\
	cd ../../../../scratch/fall2017 \\
	mkdir 27 \\
	cd 27 \\
	git init \\
	git clone git://git.yoctoproject.org/linux-yocto-3.19 \\
	cd linux-yocto-3.19 \\
	git checkout tags/’v3.19.2’ \\
	git remote set-url origin https://github.com/sharma-akash/OS2 \\
	cd ../../../files \\
	source environment-setup-i586-poky-linux.csh \\
	cp bzImage-qemux86.bin ../fall2017/27 \\
	cp core-image-lsb-sdk-qemux86.ext4 ../fall2017/27 \\
	cp config-3.19.2-yocto-standard../fall2017/27/linux-yocto-3.19 \\
	cd ../fall2017/27/linux-yocto-3.19 \\
	mv config-3.19.2-yocto-standard .config \\
	make -j4 all \\
	cd .. \\
	qemu-system-i386 -gdb tcp::5527 -S -nographic -kernel bzImage-qemux86.bin
	-drive file=core-image-lsb-sdk-qemux86.ext4,if=virtio -enable-kvm -net none
	-usb -localtime --no-reboot --append "root=/dev/vda rw console=ttyS0 debug” \\
	Open new window and connect to the server \\
	Gdb \\
	Target remote :5527 \\
	Continue \\

\section{Qemu Command-line Flags}

	\textbf{-gdb tcp::???} Initialize a GDB stub at port ??? \\
	\textbf{-S} Manually start the CPU \\
	\textbf{-nographic} No graphics for Qemu, only command line \\
	\textbf{-kernel} Set the kernel file image \\
	\textbf{-drive} Add a drive, we add from a file \\
	\textbf{-enable-kvm} Enable KVM virtualization \\
	\textbf{-net} Configure network devices, we set none \\
	\textbf{-usb} Enable usb to add usb devices \\
	\textbf{-localtime} Set what the current time is for the kernel \\
	\textbf{--no-reboot} Don't reboot, just exit \\
	\textbf{-append} defines the command line for the kernel \\

\section{Concurrency Write Up}
	1. The purpose of this assignment was to allow us to become familiar with the
qemu-based yocto environment and working with the VM. Furthermore, it allowed us
 to refresh our skills from Operating Systems 1, in terms of programming.
  Lastly, it taught us how to think about threads working concurrently and
	 how to optimize their performances without sacrificing functionality.

2. I decided to just educate myself about how the producer-consumer problem
worked theoretically so I looked at references and other tutorials. Next, I wrote
out my implementation on paper and decided to use a mutex lock to prevent to threads
 from accessing a buffer, a signal to indicate when the lock was removed, as well as
  a wait for whenever we needed to wait a random period of time.

3. To ensure my solution was correct, I tested boundary cases such as when the
 buffer was full, empty, or simply normally filled. I ran the program multiple times to ensure the random numbers were consistently in range, and also to make sure that we maintained exclusive buffer access.

4. I learned about how complex it can be when two threads are working on the same
data, concurrently. A lot of stuff goes on that you do not expect to happen so the
 control flow is quite inconsistent. Further, I learned the importance of thinking
 in parallel because without two threads, the best we could do is have a producer
  make 1 object, a consumer consumes, and the cycle continues. With our method it
	allows the threads to continuously work and produce results in a shorter time.


\section{Version Control Log}

\begin{tabular}{ c | c }
	\hline
	462092b10d2c5876be67ebd0073c2841cb94a031 &  fixing makefile \\
	\hline
	6cd536c5e0cd880cf265ccd0fa77b783cab218a0 &   Finalizing concurrency, adding latex \\
	\hline
	e9411b8a587380de32cd1fd318b9f3700964ecc5 &	Added Makefile, mersenne, some small changes \\
  \hline
  ce34cbb44cadc33ff07d5089b6b6102e416f4f1e &  removing temp file \\
	\hline
  f12ca65be464186e8661fbd79ecc4da149c93881 & initial commit for concurrency code \\
	\hline
  b1089512127b723a3247c7aba611d8a6e57d9ad7 &  Adding code\_repo \\
  \hline
\end{tabular}

\section{Work Log}

	10/5 - Created the folder for the group, pulled the code and created a repo \\
	10/6 - Built the kernel, ran the VM \\
	10/7 - Started working on concurrency code \\
	10/8 - Implemented mersenne, continued working on other parts of concurrency\\
	10/9 - Finished concurrency, wrote latex and answered questions\\

\end{document}
